%# -*- coding:utf-8 -*-
\documentclass{article}

\usepackage{fontspec,xunicode}
\setmainfont{黑体}
\usepackage[slantfont,boldfont]{xeCJK}
\usepackage{xcolor}
\usepackage{amsmath}
\setCJKmainfont{黑体}
\usepackage[final,xetex]{graphicx}
\usepackage{float}
\usepackage[margin=3.5cm]{geometry}
\usepackage{listings}

\begin{document}

\lstset { 
numbers=left,
numberstyle=\footnotesize,
numbersep=5pt,
backgroundcolor=\color{lightgray}
}

\author{陈睿, 方宇剑}
\title{Ucore Go Porting实验文档}
\maketitle

\section{实验目的}
完成GO到UCORE运行时的移植,使得编译好的GO程序能够在UCORE上运行。

\section{编译ucore-go}
直接运行根目录的
\begin{lstlisting}
demo.sh
\end{lstlisting}
等待几分钟(根据机器性能而不同),就能完成所有的编译操作。此后就能直接开启ucore,进行相应的go程序测试了。

demo.sh基本上完成了如下步骤的操作:
\begin{enumerate}
  \item 完整编译在Linux上的GO编译器;
  \item 将Patch替换原有的GO语言相应包,使得GO编译器源代码具有UCORE的运行时;
  \item 部分编译替换的内容(如runtime、os等)。至此,该GO编译器已经是宿主为Linux,目标为Ucore的交叉编译器;
  \item 利用编译好的交叉编译器对testsuit内的测试go源程序进行编译。其中如果是编译错误、特殊错误或者是未能通过的测试样例被剔除出来,另行人工处理;
  \item 将编译好的GO可执行文件拷贝到Ucore系统的相应文件夹,使得这些文件能在Ucore里出现;
  \item 编译修改过的Ucore系统,以及相应的磁盘文件disk0;
  \item 启动Ucore系统,可以在磁盘中找到编译好的Go程序并运行了。
\end{enumerate}

\section{文件修改集}
由于GO的原有文件需要保留用以进行相应的Linux编译器支持(放在src/go中),因此对GO的更改放在patch文件夹中,而Ucore的更改直接在Ucore上进行,放在src/ucore中。

testsuit包含了以go语言编写的测试集,用以在Ucore上进行测试,测试Go的运行时是否正常。除了一些测试编译时错误、特殊错误和尚未在Ucore实现的特殊机制外,在使用脚本编译时,这些测试会被自动编译。

\subsection{Go相关修改集}
对于Go的更改分为两种。一种是确实进行了相应的增减,一种是直接从Linux处复制而来。直接从Linux处复制而来的有如下一些文件,其中可能有冗余的文件,亦即删除这些文件也能使得GO正常工作:
\begin{lstlisting}
patch/src/pkg/runtime/cgo/ucore_386.c
patch/src/pkg/runtime/ucore/defs.c
patch/src/pkg/runtime/ucore/defs1.c
patch/src/pkg/runtime/ucore/defs2.c
patch/src/pkg/runtime/ucore/defs_arm.c
patch/src/pkg/runtime/ucore/signals.h
\end{lstlisting}

其余的文件,都针对Ucore进行一定的更改工作。

\subsection{Ucore相关修改集}

\begin{lstlisting}
kern/mm/pmm.c                        // ldt
kern/mm/pmm.h                        // ldt
kern/mm/memlayout.h                  // ldt
kern/process/proc.c                  // ldt mmap exit\_group
kern/process/proc.h                  // ldt
kern/mm/vmm.c                        // mmap
kern/mm/vmm.h                        // mmap
kern/syscall/syscall.c               // ldt mmap gettime
kern/syscall/syscall.h               // ldt mmap gettime
libs/unistd.h
libs/types.h
Makefile                             // gcc46
ucore.lds                            // gcc46
\end{lstlisting}

\section{实验过程详述}

我们着手进行的第一件事情,就是对Go源代码结构进行分析,以确定我们需要修改的部分。经过分析可以了解到,Go语言的基本支持库作为Go的一个package放在/src/pkg文件夹中,命名为runtime,所有与操作系统相关的基础运行时支持都包含在这一文件夹中。因此,只需要在这一文件夹中进行修改(添加ucore目录),再调整相应工具链,就能完成Go语言在Ucore上的支持。除此之外,我们还发现,出于工程上的考虑,go语言除了runtime之外还有一个原生的syscall接口,允许go语言直接进行系统调用。除此之外,通过搜索,还有os和net两个包中包含c或goc源文件(亦即操作系统相关)。网络层面的支持将暂缓,因此net包中暂时搁置。于是,最终我们敲定更改的包,集中在runtime、syscall和os三个包上。

以下是我们进行大工作详述,大致按照时间顺序排列。

\subsection{建立hg版本控制,调整文件结构}
为了在两人之间进行统一的版本控制管理,我们决定使用Mercurial来进行管理。因此,第一步就是建立起文件目录结构,并在这上面建立版本控制库。

\subsection{使用gdb在ucore上进行用户态程序的调试}
由于gdb实现的是在qemu上进行内核态调试的功能,如何在ucore上进行用户态程序的调处,如何让调试器跳过所有内核初始化操作,设置在用户态程序上的断点,是我们需要考虑和探索的问题。最终经过实际测试,将go的可执行文件反汇编的结果,其中的指令地址就是实际执行的地址。因此,将断点设置为这些地址,则可以直接在gdb中进行调试(从来没有出现过断点失败的情况)。因此,我们最终仍然使用与内核态相同的调试方法进行用户态程序调试。

\subsection{在Linux上进行go的ucore交叉编译}
由于我们的Go编译器仍然需要在Linux上运行,如何实现交叉编译,仍然是我们需要考虑的问题。一开始,我们认为Go似乎自带了交叉编译的支持,只需要修改Makefile就能完成相应的交叉编译器的编译。但是我们最终发现,不仅实现起来需要修改很多的Makefile,其中还有Go特有的Makefile格式,而且Go本身也并没有对交叉编译有一个很清晰的支持,我们还需要想其他的办法来实现交叉编译。

最终,我们从Go的Tiny运行时中得到了灵感。Tiny运行时是运行在裸机上的Go运行时,能够使得Go程序不需要操作系统就能运行起来。因此,对Tiny来说,交叉编译是必须的。他们的做法是,先对Go编译器进行一次全编译,然后将Tiny的相关patch打给源代码,再对更改的部分进行一次部分编译。这样做完以后,编译器就是交叉编译的了。我们依次类推,进行了相应的交叉编译工作。经过测试以后,编译通过,在后面的测试也可以看到,编译的程序是能够在Ucore上正确运行的。

\subsection{建立编译工具链}
由于每一次的更改都需要对Ucore和Go分别重新编译,并互相进行一些文件拷贝操作,多次进行不仅容易出错,工作量也大。为此,我们分别建立了编译工具链,在各自不同的机器上进行运行。

\subsection{最大化工作集}
GO的runtime文件中,已经进行了有效的组织。与操作系统无关的源文件被放在runtime文件夹下,而每个操作系统或是系统架构相关的代码放在独立的文件夹下(如386、windows)。但是即便如此,为了确实确认我们需要完成的工作,我们首先将所有的runtime代码过了一遍,理清函数调用链和其中的依赖关系,确认我们的最大工作集是哪些函数。工作集在Wiki上的Week4有详细列表。

\subsection{syscall包的支持}
如前所述,我们需要对go的syscall包提供相应的系统移植。由于这是一个最原始的系统调用接口,基本上,我们只需要提供系统的调用号就能完成syscall包的编写。

\subsection{分析Hello World程序的需求}
由于成功编译并运行Hello World程序需要一个最小的运行时支持,是能够运行起来的最简单的程序之一;另一方面又包含了相对完整的运行时支持,例如打印的支持对后续的调试提供便利。因此,使得Hello World在UCore上运行起来成为了我们的第一件工作。为此,我们需要分析Hello World可能需要我们实现的运行时功能。

我们最初认为,实现Hello World只需要实现输出相关的函数。但是经过实际调试发现,在此之前,我们需要完成建立LDT表以及一系列内存进程相关的功能,设置好一个正常Go程序所需的运行环境。由于ldt是我们在单步调试中遇到的第一个功能,因此我们决定在此后首先进行ldt相关的工作尝试。

\subsection{最小化工作集}
先前的“最大化工作集”主要目的在于最大程度上剥离操作系统相关部分,而最小化工作集则是为了逐步缩小这个范围,从而缩小到我们需要关注并修改的代码。最小化工作集的结果是将我们的工作集定位在三个包上:8l, syscall, runtime。8l的作用是设置最后可执行程序的链接方法,而我们的工作也只是将它简单的设置成ELF32格式。runtime是我们首先需要实现的包,它是整个GO得以运行的基础,也可以认为是第一个默认的m可以运行的基础。在GO的层次结构中,可以将其视为最底层。syscall提供了调用syscall的接口,它是更多上层包的基础。多大数上层包向下接洽都会使用syscall包,比如fmt包专门管理格式输出,它会调用syscall的Write函数来满足最底层的支持。

\subsection{LDT的第一次尝试:建立syscall框架}
为了设置LDT,我们仿造Linux的做法,在ucore中新增一个modify\_ldt的syscall来实现这一目的。同时,我们也可以进一步熟悉ucore的框架,为未来添加syscall做一定的尝试。最终,我们搭建了一个证明可行的框架,因为根据调试结果来看,系统确实运行到了相应的系统调用中。

\subsection{LDT的第二次尝试:Fake LDT by GDT}
由于ldt这一概念比较少见,我们很难找到与之相应的实现。在对相关文档做了一定的研究之后,我们提出了两种在ucore上实现ldt的方法:一种是与modify\_ldt一致,使用CPU的lldt指令来装载ldt,这样做能够很好地满足go的需要,但是我们需要重写一套相应的实现,并且引入ldt概念可能会引起ucore中更多的未知错误;另一种是仿造go的tiny实现方法,在gdt中新增一条目,在装载ldt时直接将offset和limit等数据装载到gdt中,达到伪装的目的。这一种方法可能实现起来略显粗糙,但是对现有框架的改动不大,更容易实现。因此,我们最终首先尝试了第二种方法,经过一些调试,Hello World成功通过了setldt阶段。

\subsection{runtime.rt\_sigaction}
在此之后,Hello World进入rt\_sigaction函数。这是一个与信号设置相关的函数,由于当时ucore并没有相应的signal支持,直接忽略该函数的函数体返回即可。

\subsection{runtime.mmap}
接下来,Hello World通过一些操作系统无关的控制函数,来到了内存分配的阶段。通过调研Linux的相关实现,我们发现所有的Go语言内存操作都是基于Linux的系统调用mmap来实现了,而这一调用在ucore已经有一个基础的实现。具体来讲,ucore中的mmap并不支持文件映射,但是对于go的runtime来说,这已经够用。因此,我们进行了相应的更改调用号、调整传参顺序的更改。

此外,linux的mmap是返回分配的内存地址,而ucore中是在第一个参数中传入指针的指针进行值的返回,我们也在go的runtime中做了相应的调整。

\subsection{runtime.exit1}
仿造Linux的exit,我们更改了调用号,调整了传参顺序,实现了相应的ucore版本。

\subsection{runtime.gettime的第一次尝试}

\subsection{runtime.write的第一次尝试:循环putc}
由于我们对于write的第一次尝试(更改调用号,调正传参顺序)没有成功输出字符,我们便开始考虑其他种write的实现方法。通过ucore的putc系统调用循环输出字符似乎是一个不错的方法,我们也进行了相关实现。这一次,ucore成功地打印出了相应的字符,但是输出字符并不是Hello World。

\subsection{runtime.write的第二次尝试:重定向stderr}
通过对runtime.write的进一步分析,我们发现原始的printf在调用runtime.write时,会试图通过write向stderr写入字符。如果将其改为stdout会如何?于是我们放弃putc的思路,改回原先的版本,忽略了传入的参数,无论什么情况都向stdout写入。这一次,虽然仍然会在之后出现Page Fault,屏幕上终于出现了正确的Hello World,这是因为ucore还没有stderr的相应逻辑。为了使得runtime.write更加通用,我们添加判断,只将stderr的字符转向stdout输出。runtime.write至此完成。

\subsection{ucore的mmap修改:分配最近空闲内存, 增加新的mmap辅助函数get\_unmapped\_area\_with\_hint}
实际上,mmap系统调用在Linux上会传入一个地址,但这个地址实际上并不是一个硬性要求,而只是一个“hint”。ucore中获取空闲内存从高地址往前搜索,因此并没有“hint”

\subsection{go的内存运行时:SysMap, SysReserve, SysAlloc, SysFree, SysUnused}
GO语言提供了完整的原生垃圾回收支持,但是将代码分析之后,可以看到与操作系统相关的函数无外乎以下五种:SysMap, SysReserve, SysAlloc, SysFree和SysUnused。只要完成了这五个函数的正确实现,GO语言的垃圾回收机制、自治堆栈分配等应当都能正确地运行起来。

我们最初根据linux的实现按照我们的理解对mmap的调用进行了修改,使之符合ucore的约定,但是在运行中throw的情况。GO语言的throw通常是因为其自检没有通过,这是为操作系统开发者提供的一种提示方法,能够帮助操作系统移植者进行代码的检查。这些Throw提示我们,Arena的分配不正确。

Arena是GO语言从操作系统处申请预留的地方,在一些情况下出现内存分配的时候,GO会尝试在Arena中开辟内存空间。同时,GO自己维护Arena中的占用情况,在Arena被占满时会提示进行相应处理。但是我们的Throw和一系列printf语句表明,在Arena上分配的内存空间最终并没有在Arena上,该地址实际超出了Arena的范围。

经过对GO语言源代码的一些分析,我们发现,Arena部分的空间是使用SysReserve分配的,而从Arena中分配是通过SysMap分配的。这二者之间究竟有什么区别呢?他们和SysAlloc又有什么联系呢?带着这些问题,我们又看了一遍Linux的实现,终于发现了问题所在。Linux中对于这三种情况的处理,mmap的调用是基本一致的,只有在SysMap中存在着特别的地方。在SysMap中,mmap的调用是带有MAP\_FIXED标记的。这个标记的解释是,无论这块内存是否已经分配,将强制再将这块内存分配给进程。

至此,我们才明白这三种方法之间的联系和区别,这种关系实际上通过名字是可以看出来的。SysAlloc就是最普通的内存分配,没有太大的特殊性。而SysReserve则是保留一块内存区间以待未来取用,SysMap则是在曾经Reserve的空间中进行取用。通过这样两层的方式,我们就可以保证这样一段数据在内存中是线性存放的。虽然说将二者区分开来,有利于在操作系统层面上做相应的优化,但是Linux上只是直接将其进行了分配。

由于ucore中的mmap并没有相应的MAP\_FIXED功能,因此在具体实现上,我们使用了在Reserve时真分配,Map时假分配的方法。只要保证SysMap的地址是正确的(通过GO的上层保证),在SysMap中直接返回就能达到我们的要求。经过相应的更改后,终于通过了这一条throw语句。

\subsection{MAP\_ANON标志清零内存}
在此之后,我们又遇到了Page Fault的问题。从目前来看,能够通过前面的内存设置检查的throw语句,Page Fault的问题很可能是由指针引起的。果不其然,找到Page Fault处的IP,发现这是一个在取地址时抛出的错误。GO语言支持多层次的内存结构,其中MCache用来分配较小的内存,不需要重新请求内存,提升了性能,而Page Fault正是在MCache\_Alloc中,访问MCache结构指针c的\&c->list[sizeclass]时发生的。经过代码的搜查,发现MCache的list在初始化时(malloc.goc中的runtime.allocmcache),并没有相关置零的操作,在其中添加了相应的操作以后调试通过该Page Fault。

但是,为什么在Linux上的程序没有类似的问题呢?这又涉及到了另一个mmap标志符MAP\_ANON。MAP\_ANON的主要作用,在于表示这一次的mmap不映射任何内存,同时将这段内存清零。因此,只要在SysMap等函数中额外使用memclr做一次清零的工作,就不需要在其他地方修改代码了。

\subsection{添加链接器8l对Ucore的支持}

\subsection{屏蔽环境变量}
在接下来的调试中,同样的Page Fault问题又再一次出现了。这一次的指令是出现在findnull函数中。这个函数非常简单,提示我们从上一层函数中寻找问题。但是由于GO打印Traceback的相关操作并不是操作系统无关的,我们并没有实现相应的Traceback打印,因此我们不得不进行一次dirty的手工Traceback。基本思路是,在单步过程中,只要跳过某一个函数时发生了Page Fault,就跳进该函数继续递归执行上面的过程。最终我们找到了如下的一个调用链:runtime.goenvs\_unix -> runtime.gostringnocopy -> runtime.findnull。

于是我们意识到,由于Ucore并没有环境变量的支持,在这里寻找相应的字符串,很可能会因为找不到'\\0'而使得指针最终指向一段不被允许的内存空间。于是,将环境变量全部设置为空,系统调试通过。

\subsection{Hello World完成!}
在完成上述工作之后,就可以在ucore上正常地运行Hello World了!接下来的工作,就是测试更多的go程序,尤其是并行方面和内存方面需求较大的程序,一方面以此为指向不断完善runtime的功能支持(如gettime、clone、lock/unlock的支持),另一方面,通过这些测试来发现之前工作中潜在的漏洞。

\subsection{tls\_offset调整}
tls\_offset在ucore上与linux不同,因此在设置g和m时GO会throw报错。g和m是go用来进行线程管理的两个结构体,每个线程都有对应的g和m,后面会有详细的阐释。设置LDT,就是为了建立起TLS,存放g和m这两个结构体的指针。根据go的mkasmh.sh的描述,为了和pthread库一致,8l连接器做了一定的调整,这要求不同的操作系统根据实际情况选择g和m的偏移地址(有的直接放在TLS基地址,有的放在TLS-4的位置)。此后,GO会对其做一次存取检查,以确保TLS的正确建立,否则将throw一个异常。

由于地址是通过8l调整的,因此只要让8l不进行这个调整,ucore就可以不进行额外的调整而正常运行。上面链接器已经做了相应的调整,因此g和m已经能够进行正常的存取了。在TLS建立起来以后,多线程的程序才有了可能性,我们进一步的工作也就成为了可能。

\subsection{runtime.gettime的第二次尝试:添加系统调用}

\subsection{完善syscall包, 添加os包实现}
在runtime的工作告一段落之后,我们开始对syscall和os进行完善。我们将新的syscall加入了syscall包,进行一定的校正。此外,通过对os包的调研,我们发现在ucore系统上,并没有太多我们需要完成的工作,只需要将linux的相应实现复制一份即可。

\subsection{合并编译工具链}
在之前的工作中,我们很早开始使用自动化脚本对整个编译进行控制。自动化脚本可以重新部分编译go的runtime,重新编译ucore,将二者整合起来,再编译HelloWorld放到。最早是一人使用Python脚本,一人使用Bash脚本各自使用各自的工具链。然而,随着版本控制push和pull越来越频繁,建立统一的工具链变得更有必要了。因此,我们经过讨论决定,对Bash脚本做了一定的调整,使之通用化,使得两个人之间能够达成一致。

\subsection{添加测试集:着手并行程序实现}
我们挑选了几个go中与goroutine、channel等并行特性相关的go测试源文件,通过测试它们在ucore上的运行情况,对运行时进行调试。这些文件有:


\subsection{runtime.gettime的第三次尝试}

\subsection{clone的实现}
原有Linux的clone基本上只需要更改调用号、调整参数顺序就能实现,外加上正确的gettime就基本上能够正常工作。但是clone中间还用到了一个get\_tid的方法,这个方法的作用是在clone出新的线程以后返回其线程ID,这一线程ID被用来作为LDT表项的编号(如ID为1的线程对应第7条LDT表项,ID为2的线程对应第8条LDT表项)。在ucore中,没有相应的gettid系统调用,直接调用getpid既能正常工作。但是这同时也带来一个问题,一旦线程数量多了,LDT表项也随之增加,这是我们在单线程程序中没有考虑过的问题。

\subsection{lock/unlock的第一次尝试:Semaphore替代Futex}
进入了并行程序的调试阶段,我们开始需要实现Lock与Unlock的相应功能。在Linux中,这一部分的功能是使用了Linux的系统调用futex来实现的。futex是一种高性能的内存锁,能够有效地提升并行程序的性能。然而在ucore中,我们并没有futex的相关实现。考虑到futex实现复杂,涉及到内存、进程等诸多模块,我们最终决定参照其他操作系统的做法,使用ucore已经实现了的semaphore来进行lock与unlock的实现。为此,我们参照了darwin系统的相关实现。具体来讲,就是通过一个从0到1、从1到0的semaphore来模拟锁的过程。但是在实现完成后,一旦运行相应的程序,ucore会重新启动。经过一番排查,我们发现,在go的一个线程调度算法中,在切换线程后,程序的指针跳转到了系统启动的地方。这导致了ucore的重启,进一步原因仍然有待查明。

\subsection{sleep实现}

\subsection{究竟有几个线程?GO的线程机制goroutine探究}
Go语言的并行模型是通过一种新型的goroutine来实现的,每一个goroutine都是一个函数,代表了类似过去线程的概念,从而达到并行执行的效果。在对测试样例的实验中我们发现,即使开了很多的goroutine,在没有更改过clone和lock/unlock的情况下,程序也能正常执行(例如每个goroutine打印一个素数)。最后我们发现,在这种情况下,程序实际上根本没有向操作系统要求更多的线程,而是完全串行化处理的。为什么程序并没有并行化?虽然这个程序前后有依赖的要求,难道go中包含了什么机制来判断是否需要新开线程吗?

带着这种疑问,我们用上了刚刚实现的sleep函数。这一回,程序确实由于调用了不正确的clone和lock/unlock而导致了崩溃。由于sleep函数使得线程之间的顺序关系出现了变化,因此程序不再是串行化的,也就有了新开线程的必要。

经过一番资料的调研,我们了解到,go的线程实际上包含了两层的模型,也就是之前提到过的g和m。一个m结构描述了一个操作系统层面的线程,由操作系统管理;而一个g结构描述了一个go语言层面的轻量级线程,由go自行管理。每一个g对应了一个m,而一个m可能对应很多g。当系统发现现有的m不能满足要求时(如内存方面的要求,一个m所能拥有的最大g数量等),就会通过clone来向操作系统申请新的m。这种结构也很好地解释了我们之前所遇到的情况。因此,根据这些调研,在完成了相应的系统调用之后,所有的并行go测试应该都能正常通过。

\subsection{改善工具链1:加入usage信息}
为了未来能够将工作转移给他人进行,在工具链脚本中加入帮助信息是很有必要的。因此,我们为脚本添加了usage信息,指示用户如何进行部分编译的相关定制化操作。

\subsection{lock/unlock的第二次尝试:不会重启了}
由于通过排查,我们确实无法找到lock/unlock的问题。因此我们对其重新进行了一次实现,同样参照Darwin的实现。但是这一次,lock/unlock竟然不会重启了。我们对之前的程序进行单步调试,发现当程序运行到某一步,会出现地址平移的情况,本应跳转的正确地址加上了一个相当大的偏移,致使系统跳转到了不正确的地方而重启。我们推测,可能是由于某些基本的操作在第一次的时候编码错误,使得系统重启。

\subsection{进程保存FS和GS寄存器}
在前面的锁完成以后,我们又遇到了其他的问题。同样是在多个m发生的时候,在切换进程时,go语言会throw报错:bad m->nextg in nextgoroutine。通过查看源代码,我们在proc.c找到了如下代码:
\begin{lstlisting}{lagnuage=C}
        m->nextg = nil;
        m->waitnextg = 1;
        runtime·noteclear(&m->havenextg);
        if(runtime·sched.waitstop && runtime·sched.mcpu <= runtime·sched.mcpumax) {
                runtime·sched.waitstop = 0;
                runtime·notewakeup(&runtime·sched.stopped);
        }
        schedunlock();

        runtime·notesleep(&m->havenextg);
        if((gp = m->nextg) == nil)
                runtime·throw("bad m->nextg in nextgoroutine");
\end{lstlisting}
可以看到,m->nextg应该指的是一个m线程上下一个要执行的g线程。代码首先将m->nextg置为nil,之后又判断要求其不等于nil,这意味着中间的代码对m->nextg进行了更改,但是中间这些函数并没有修改m的相关代码。经过一番困惑后,我们突然意识到,这意味着这一段代码并不是串行执行的,在中间的noteclear与notewakeup函数与锁有很大的联系,很可能是进行到中间,代码切换到了另外一个进(线)程中,才使得这段代码正常工作。

那么,究竟是什么样的原因,使得切换后的线程m没有正确赋值,或是没有正确切换呢?我们首先想到了lock/unlock的实现正确性。为此,我们额外编写了一个测试semtest5.c,用以模拟go中lock与unlock的全部过程。通过测试我们发现其运行正常,鉴于目前没有看到lock/unlock潜在的问题,并且也通过了go的一些启动自检,我们只能初步认为,问题发生在lock/unlock之外。

就在我们一筹莫展时,对于这段代码的调试又有了新的进展。我们对这段代码进行了大量的调试输出,看到了这样一种情况。假设现在有两个m线程:线程4和线程5。线程5刚刚运行完,正要通过schedule切换到线程4,因此通过noteclear和notewakeup来使得线程5睡着,睡眠中的线程4被唤醒,那么这个时候,所有的工作就应该切换到线程4进行。但是最后在throw的时候,当我们将线程号打印出来的时候,我们发现,这个throw仍然在线程5中发生。这意味着由于某种原因,m结构仍然是线程5的m结构。我们根据这一个现象,最终想到了GS和FS寄存器。最早在通过LDT设置TLS的时候,我们将TLS的段地址放在GS中,而TLS中存放了g和m的指针,这意味着GS决定了取到的g和m内容。接着我们又发现,Ucore处于简洁考虑,在进程切换时并没有切换GS和FS寄存器。这就使得GS仍然是线程5的GS,m也就是线程5的m了。于是,我们在Ucore将GS和FS保存在进程结构中,切换时予以切换,终于调试通过。

\subsection{LDT的第三次尝试:好多好多的GDT}
如前所述,当出现多个m线程时,调用clone并设置多项LDT条目成为必备的功能,而这些功能在单线程程序中是不存在的。因此,我们必须提供设置多个LDT条目的可能性。由于之前我们使用了在GDT中提供Fake LDT的方法,现在妥协的做法是在GDT中添加多个表项。由于测试程序最大开到3个m线程,因此当我们GDT设置到大于3个条目时,就应该能使测试程序正常工作。经过这一过程后,实测测试程序通过。这一方法只是权宜之计,但是证明了我们的clone函数以及一些相关函数能够正常工作,方便我们进一步展开工作。

\subsection{lock/unlock的第三次尝试:修改等待超时}

\subsection{输出混乱,是Ucore的问题吗?println的原子性保证}
在完成了lock和unlock的相应实现之后,我们遇到了与打印相关的问题。我们的测试程序利用n个goroutine进行n次打印“Hi”,因此当n=5时,屏幕上预期得到“HiHiHiHiHi”的结果,这一点在Linux上得到了验证。可是在Ucore上,使用系统底层的printf,可以得到正确的结果,但是在使用fmt.Println这一函数时,我们得到了“HHHHHiiiii”的结果。一开始我们认为是lock/unlock的问题,但是其他类似素数打印的程序能够正常工作,又让我们疑惑不解。

最终,我们突然意识到,在go语言的逻辑层面,println并没有原子性的保证。底层的printf可能是由操作系统来保证原子性,但是这并不意味着在并行模型下,这一原子性的必须的。因此,虽然在Linux上能够正常输出,但是在Ucore上的不正常输出也并不代表程序的实现错误。如果go希望能得到通用正确的结果,应该使用一个输出锁,在每次打印之前获得该锁,而不是由操作系统来完成相应的职责。

\subsection{exit\_group的第一次尝试:runtime.exit实现}

\subsection{LDT的第四次尝试:完美解决,单个GDT,进(线)程保存独立LDT}
前面提到我们使用多个GDT条目模拟LDT的方法来测试并行程序,但是如果go程序要求开成千上百个程序的时候,我们原先的方法就力不从心了。为了彻底解决这一问题,我们提出了新的方法。每时每刻,由于只有一个m线程在工作,实际上也就只有一个LDT在工作,那么我们为何不只设置一个GDT表项,在进程切换时将LDT替换进来即可呢?经过相应的尝试,我们发现这一做法能够很好地工作,并可以开启任意数量的线程。同时,由于使用了系统调用modify\_ldt,我们实际上隐藏了Ucore的具体实现,因此在未来利用这一框架替换成LDT也是可能的。

\subsection{改善工具链2:提供测试go功能}
随着需要测试的go测试集越来越大,提供自动化测试脚本的需求越来越高。为此,我们在原来的工具链脚本中加入了测试的相应功能,实现了一个脚本完成所有功能的效果,并能够进行定制,单独进行某一部分的编译或测试。

\subsection{exit\_group的第二次尝试:从timer\_list中移除proc}

\subsection{GO的测试用例测试}

\subsection{GCC 4.6的UCore编译支持}
在系统升级到GCC 4.6后,我们意外地发现Ucore不能编译成功了。原因是编译出来的bootblock大于510字节,不能被完整地放到第一扇区中。经过测试,回滚到GCC 4.4是能够正常编译的,但是为了能在未来使用新的编译器,我们还是决定尝试探寻在GCC 4.6上编译Ucore的方法。

经过一番尝试和向同学们询问,我们了解到,新的bootblock多出来的那部分,是由于增加了.eh\_frame所致的。.eh\_frame是用来进行异常处理的相关代码段,在C中并不是必须的,在操作系统的bootblock层面就更没有用处了。GCC 4.4默认不会添加这个代码段,但是到GCC 4.6中情形不一样了。根据同学们的说法,使用strip命令可以将.eh\_frame去掉,实际结果确实如此,但是strip命令同时也将目标文件的所有符号都去掉了,这使得bootblock在切换为保护模式时跳转到了错误的地址,因此编译完成后Ucore不能正常启动。

为了解决这一问题,我们查找了一些相关论坛,最终综合几种方法,决定通过更改链接器脚本来去除.eh\_frame。链接器脚本中定义了链接器进行链接时的操作。我们通过指示链接器不将.eh\_frame链接进来的方法,完成了bootblock的链接。为此,我们修改了Makefile,并添加了独立的链接脚本ucore.lds。经过实际测试,上述方法能够在GCC 4.6上编译UCore成功。

\subsection{有问题的用例}
nilptr测试集顾名思义,是用来测试go对于不合法指针的正确报错处理的测试集。在实际测试中,这一测试集并没有通过相应的测试。我们注意到,测试要求这些go程序throw出非法地址的异常,而这些异常是通过signal由操作系统传递给go程序的。因此,在signal实现之前,这一测试集还不能正常通过。

同时,由于环境变量没有实现,在环境变量测试时也会出现一些问题。如果为系统设置了相应环境变量,而这些环境变量却不能被读出来,那么这些测试自然会报错。

\section{未来的工作}

\subsection{添加SIGNAL实现}
虽然没有Signal,Go程序也已经能够很好地在Ucore上运行,但是如果出现需要Ctrl-C结束程序的情况,或者nilptr中指针地址异常的情况,Signal机制能够有效地通知程序相应的异常。因此,未来的工作应当是在Ucore中添加Signal的底层支持,同时按照POSIX标准实现相应的各种Signal情况。

\subsection{添加环境变量支持}
之前遇到环境变量时,我们直接对其做了屏蔽处理,使得一些Go的测试不能通过。因此,能够在Ucore中实现环境变量的支持,就能够使得Go的支持更加全面。

\subsection{添加syscall: kill}

\subsection{操作系统对GO的并行特性的原生支持}
虽然Go语言引入了很多新的并行语言特性,如Channel、Goroutine等,但是一旦深入的底层我们就会发现,Go语言实际上在操作系统层面,仍然使用了原有的锁这些并行编程的概念。从现实的角度来讲,这是对现有操作系统的一种妥协,本无可厚非。但是如果想要进一步挖掘Go语言的潜力,我们可以尝试从操作系统层面提供一些例如channel相关的接口,进一步提升go语言的性能。

但是从实现上来讲,这就不仅仅涉及到运行时的一些系统调用的简单问题了,进行原生支持,将会涉及到g和m的更改,涉及到大量源代码的变动。同时,从性能的角度来讲,能够提升多少尚未可知。因此,这样的工作是否值得,仍有待进一步调研。

\section{参考内容}

交叉编译参考了Tiny的方法,实验主要参考了Go的Tiny运行时和Linux运行时,Semaphore部分参考了Darwin系统的Runtime。

\section{致谢}
实验中向勇老师和陈渝老师提供了很多的帮助(和督促),在此表示感谢;

王乃峥师兄也为我们提供了很多帮助,有一次还特意和我们见面详谈,在此表示感谢;

开发GO语言的×××和×××很热心地回答我们的问题,提供了不少帮助,在此表示感谢。

朱文雷、沈彤、张超等也曾经为我们小组答疑解惑,在此一并感谢。


\end{document}
